\documentclass[a4paper,12pt]{extreport}
\usepackage{preamble}

\begin{document}
  \begin{center}
    \textbf{Дальневосточный федеральный университет}\par
    Школа естественных наук\par
    Домашнаяя работа по предмету <<Логическое программирование>>\par
    (8-й семестр, прикладная математика и информатика)\par
    \textbf{группа № Б8403А, Федоров Константин Сергеевич}\par
    \textit{28.02.2018}
  \end{center}

  \chapter{Реализовать метод $fib(N, F)$, где $F$ - это $N^{\text{\underline{e}}}$
  число Фибоначчи.}\par
  \begin{lstlisting}[caption=fib(N{,} F) realization, style=customProlog]
fib(0, 0).
fib(1, 1).
fib(N, F) :-
  N > 0,
  N1 is N - 1,
  N2 is N - 2,
  fib(N1, F1),
  fib(N2, F2),
  F is F1 + F2.
  \end{lstlisting}
  \section{Синтетические тесты:}\par
  \begin{lstlisting}[caption=fib(N{,} F) realization, style=customProlog]
test1 :-
  fib(0, 0). % true
test2 :-
  fib(1, 1). % true
test3 :-
  fib(2, 1). % true
test4 :-
  fib(3, 2). % true
test5 :-
  fib(4, 3). % true
test6 :-
  fib(5, 5). % true
test7 :-
  fib(6, 8). % true
test8 :-
  fib(7, 13). % true
  \end{lstlisting}
  \chapter{Реализовать функцию $gcd(X, Y, Z)$, где $Z$ - это НОД$(X, Y)$.}\par
  \begin{lstlisting}[caption=gcd(X{,} Y{,} Z) realization, style=customProlog]
gcd(0, X, X) :- X > 0.
gcd(X, 0, X) :- X > 0.
gcd(X, X, X) :- X > 0.
gcd(M, N, X) :-
  M < N,
  Q is N - M,
  gcd(M, Q, X).
gcd(M, N, X) :-
  M > N,
  gcd(N, M, X).
  \end{lstlisting}
  \section{Синтетические тесты:}\par
  \begin{lstlisting}[caption=gcd(X{,} Y{,} Z) F) tests, style=customProlog]
test1 :-
  not(gcd(0, 0, 0)). % true
test2 :-
  gcd(0, 1, 1). % true
test3 :-
  gcd(1, 0, 1). % true
test4 :-
  gcd(1, 0, 1). % true
test5 :-
  gcd(5, 10, 5). % true
test6 :-
  gcd(4, 456, 4). % true
test7 :-
  gcd(256, 856, 8). % true
  \end{lstlisting}
  \chapter{Реализовать функцию $merge\_sort(L, SL)$, где $SL$ - это отсортированный список $L$.}\par
  \begin{lstlisting}[caption=merge\_sort(L{,} SL) realization, style=customProlog]
mergesort([], []).
mergesort([H], [H]).
mergesort(L, SL) :-
  length(L, N),
  M is N // 2,
  left_sublist(L, M, H),
  right_sublist(L, M, T),
  mergesort(H, L1),
  mergesort(T, L2),
  merge(L1, L2, SL).
% getter of the left sublist [helper]
left_sublist(_, 0, []).
left_sublist([H|T1], N, [H|T2]) :-
  N1 is N - 1,
  left_sublist(T1, N1, T2).
% getter of the right sublist [helper]
right_sublist(T, 0, T).
right_sublist([_|T], N, L) :-
  N1 is N - 1,
  right_sublist(T, N1, L).
% merge of two sublist to one with sort [helper]
merge(L, [], L).
merge([], L, L).
merge([H1|T1], [H|T2], [H|T]) :-
  H < H1,
  merge([H1|T1], T2, T).
merge([H|T1], [H1|T2], [H|T]) :-
  H =< H1,
  merge(T1, [H1|T2], T).
  \end{lstlisting}
  \section{Синтетические тесты:}\par
  \begin{lstlisting}[caption=merge\_sort(L{,} SL) tests, style=customProlog]
test1 :-
  mergesort([], []). % true
test2 :-
  mergesort([1], [1]). % true
test3 :-
  mergesort([1, 2], [1, 2]). % true
test4 :-
  mergesort([3, 1, 2], [1, 2, 3]). % true
test5 :-
  mergesort([3, 4, 3], [3, 3, 4]). % true
test6 :-
  mergesort([2, 1, 10, 5, 3, 7, 6, 4, 8, 9], [1, 2, 3, 4, 5, 6, 7, 8, 9, 10]). % true
  \end{lstlisting}
\end{document}
